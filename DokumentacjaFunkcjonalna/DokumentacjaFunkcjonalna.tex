\documentclass{article}
\usepackage[T1]{fontenc}
\usepackage{listings}

\title {Maze Master \\[1ex] \large Dokumentacja funkcjonalna}
\date{\today}
\author{Jędrzej Gołaszewski i Szymon Stasiak}

\newcommand*{\subtext}[1]{\normalsize{\normalfont{#1}}}
\newcommand*{\param}[1]{[-\texttt{#1}]}
\newcommand*{\params}[2]{[-\texttt{#1}|- -\texttt{#2}]}

\begin{document}
    \pagenumbering{gobble}
    \maketitle
    \newpage
    \pagenumbering{arabic}
    \section{Opis główny}
    \subsection{Nazwa programu \subtext{
        Maze master
    }} 
    \subsection{Opis problemu \subtext{
        Problemem, który rozwiązuje program jest znalezienie optymalnej trasy
        pomiędzy punktem początkowym a końcowym w labiryncie. Program musi 
        przeanalizować strukturę labiryntu, zidentyfikować przeszkody i
        wyznaczyć najkrótszą ścieżkę, umożliwiając efektywne nawigowanie od
        punktu startowego do celu. To narzędzie stworzone jest w celu automatyzacji 
        procesu rozwiązywania labiryntów, eliminując konieczność ręcznego
        poszukiwania optymalnej drogi i umożliwiając szybkie, skuteczne 
        rozwiązanie tego zadania
    }}
    \subsection{Użytkownik docelowy \subtext{
        Użytkownikiem docelowym programu jest osoba poszukująca rozwiązania
        labiryntu najkródszą ścieżką. Program może mieć zastosowanie w badaniu
        algorytmów nawigacyjnych, grach komputerowych czy rozwiązywania łamigłówek
    }}

    \section{Opis funkcjonalności}
    \subsection{Jak korzystać z programu \subtext{
        Aby skorzystać z programu należy wpisać komendę: 
    } \texttt{./MazeMaster [argumenty programu]}}
    \subsection{Uruchamianie programu \subtext{
        Lista parametrów programu:
    }}
    \subsubsection{
        Dane wejściowe \params{i}{in} \subtext{Podajemy nazwę pliku z labiryntem jeśli jest w katalogu in bądź pełną ścieżkępliku}
    }
    \subsubsection{
        Dane wyjściowe \params{o}{out} \subtext{Podajemy nazwę pliku który będzie zawierał nasze dane wyjściowe}
    }
    \subsubsection{
        Pomoc \params{h}{help} \subtext{Flagi pomocy (wpisanie ich spowoduje że program wypisze pomoc ale nie uruchomi się)}
    }
    \subsubsection{
        Verbose \params{v}{verbose} \subtext{Flagi które sprawią że program będzie wypisywał więcej informacji które mogą byćprzydatne}
    }
    \subsubsection{
        Debug \params{d}{debug} \subtext{Flagi które sprawią że program będzie wypisywał informację potrzebne do debugowania}
    }
    \subsection{Możliwości programu:}
    \subsubsection{Poprawnie wczytanie danych labiryntu}
    \subsubsection{Zapsisanie danych wyjściowych w czytelnym formacie}
    \subsubsection{Znalezienie najkródszej drogi z miejsca A do B}
    \subsubsection{Znalezienie najkródszego wyjścia z labiryntu}

    \newpage

  \section{Format danych i struktura plików}
        \subtext{Program zakłada następującą strukturę katalogów:}
             \subsubsection{input/\subtext{ Katalog zawierający pliki wejściowe z labiryntem.}}
             \subsubsection{output/\subtext{ Katalog zawierający pliki wyjściowe z wynikami działania programu.}}
        
    
    
    \subsection{Dane wejściowe \subtext{
        Dane wejściowe programu są plikami tekstowymi opisującymi strukturę labiryntu. Każdy plik wejściowy zawiera informacje o labiryncie, gdzie poszczególne znaki reprezentują różne elementy labiryntu, takie jak ściany, wolne przestrzenie, punkt startowy i punkt końcowy.
    }}
    \subsection{Dane wyjściowe \subtext{
        Dane wyjściowe programu są również plikami tekstowymi, w których zawarte są wyniki działania programu, takie jak znaleziona najkrótsza ścieżka z punktu startowego do punktu końcowego oraz ewentualne dodatkowe informacje diagnostyczne.
    }}

    
    \section{Scenariusz działania programu }
    \subtext{Przedstawiony poniżej jest ogólny scenariusz działania programu:}
        \begin{enumerate}
            \item \subtext{Użytkownik uruchamia program z odpowiednimi parametrami, wskazując plik wejściowy z labiryntem.}
            \item \subtext{Program wczytuje dane z pliku wejściowego, analizuje strukturę labiryntu oraz znajduje najkrótszą ścieżkę z punktu startowego do punktu końcowego.}
            \item \subtext{Program zapisuje wyniki działania do pliku wyjściowego w czytelnym formacie.}
            \item \subtext{Opcjonalnie, jeśli użytkownik wybrał tryb debugowania lub verbose, program wypisuje dodatkowe informacje diagnostyczne.}
        \end{enumerate}
    

    \section{Testowanie}
  \subsection{Ogólny przebieg testowania}
Testowanie programu będzie przeprowadzone z wykorzystaniem odpowiednich narzędzi, takich jak JUnit dla testów jednostkowych oraz testowanie ręczne podczas tworzenia aplikacji. Ważnym elementem testowania będzie także zapewnienie odporności programu na różne sytuacje wyjątkowe oraz błędy, aby zapewnić jego stabilność i niezawodność w działaniu.
\end{document}
